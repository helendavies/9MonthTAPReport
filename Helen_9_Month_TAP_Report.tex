\documentclass[11pt, oneside]{article}   	% use "amsart" instead of "article" for AMSLaTeX format
\usepackage[margin=0.75in]{geometry}                		% See geometry.pdf to learn the layout options. There are lots.
\geometry{letterpaper}                   		% ... or a4paper or a5paper or ... 
%\geometry{landscape}                		% Activate for rotated page geometry
%\usepackage[parfill]{parskip}    		% Activate to begin paragraphs with an empty line rather than an indent
\usepackage{graphicx}				% Use pdf, png, jpg, or eps§ with pdflatex; use eps in DVI mode
								% TeX will automatically convert eps --> pdf in pdflatex		
\usepackage{amssymb}
\usepackage{hyperref}

\usepackage{color}
\usepackage{subcaption}
\usepackage{caption}
\usepackage{subcaption}
\usepackage[margin=10pt,font=small,labelfont=bf,
labelsep=endash]{caption}


\newcommand{\todo}[1]{ \textcolor{red}{\bf{To Do:} #1}}
\newcommand{\toref}[1]{ \textcolor{blue}{\bf{REFERENCE #1}}}
\newcommand{\future}[1]{ \textcolor{green}{\bf{Future Direction: #1}}}

%SetFonts

%SetFonts


\title{9 Month TAP Report}
\author{Helen Davies}
\date{}							% Activate to display a given date or no date

\begin{document}
\maketitle

\section{Low Temperature Plasmas and Wound Healing}
Wounds and their management present an issue, both in the UK and globally \cite{Posnett2008the}. 
In particular, chronic wounds present a significant socio-economic burden, with 1-2 \% of the population of developed countries expected to develop a chronic wound during their lifetime \cite{Siddiqui2010chronic}, and the estimated cost of wounds to the national health service (NHS) in 2012/13 was \pounds5 billion \cite{Guest2015health}.
Alongside this, as a direct result of having a chronic wound, patients can experience significant detrimental effects on their mental health, for example, due to pain, odour, mobility issues and stress \cite{Persoon2004leg}.
New wound treatment strategies are being sought, with aims to accelerate the wound healing process and decrease the bacterial load in the wound, to decrease the risk of infection.
One technique under investigation is low temperature plasma (LTP) treatment, which has shown promise for both reducing bacterial loads in wounds and promoting the healing process \cite{Kong2009plasma, Kramer2013suitability, Isbary2012successful, Isbary2010a}.

Low temperature plasmas (LTP) are weakly ionised plasmas, existing far from thermodynamic equilibrium.
This results in a plasma that has an overall temperature of $\approx$ 25-40$^\circ$C, which can be applied to biological samples without causing thermal damage. 
Evidence for the successful use of LTP in wound healing comes from both laboratory experiments and clinical trials.
For example, the ability of LTP to enhance keratinocyte migration and proliferation has been demonstrated in the laboratory \cite{Tipa2011plasma}, and clinically, devices such as KinPen and MicroPlaSter have shown promise when used on human patients, aiding healing without any obvious side-effects \cite{Isbary2013cold, Isbary2012successful, Isbary2010a, Bekeschus2016the}.

For the proposed mechanism of action for LTPs in both wound healing promotion and bacterial killing, please refer to my 6 month TAP report.


\subsection{Project Aims}
The overall aims of the project are to investigate the use of low temperature air plasma for wound healing applications. In particular:
\begin{enumerate}
\item Develop an air plasma source that is suitable for optical diagnostics \& treatment of biological samples
\item Characterise the plasma using optical diagnostic techniques to determine species densities, alongside the use of a global model, which will help with species densities predictions and guiding of experiments
\item Develop experiments to determine the effects of low temperature air plasma on biological samples \textit{in vitro}, for example, oxidative stress induction and wound healing promotion. Using these experiments, the difference in biological effects when plasma parameters (and, therefore, plasma composition) are changed can be investigated.
\end{enumerate}


\section{Work to Date}
\subsection{Simulations}
In the last few months I have been continuing with plasma simulations, in particular, the development of a comprehensive nitrogen chemistry set for use in GlobalKin.
GlobalKin is a global plasma chemistry model that calculates time evolved species densities given plasma power, geometry and feed gas, as shown in \textbf{figure \ref{fig:GlobalKin}} \cite{Stafford2004O2}.
For detailed information on this model, please refer to my 3 and 6 month TAP reports.

In the last TAP meeting I presented the nitrogen chemistry set, with its 31 species and over 200 reactions including electron impact reactions, heavy particle reactions and vibrational kinetics.
I also highlighted areas for improvement, in particular, vibration-vibration (V-V) and vibration-translation (V-T) reactions and rates, which has been my focus for the last few months.

\begin{figure}
\centering
\includegraphics[width=0.7\textwidth]{Figures/GlobalKin}
\caption{Diagram outlining how GlobalKin works. Firstly, ODEs for species density and electron temperature are constructed. This takes direct GlobalKin inputs specified by the user, as well as electron reaction rate coefficients, diffusion coefficients and mobilities from the Boltzmann solver. An ODE solver then solves the equations and the new species densities are fed back into the ODEs and Boltzmann solver, for new densities to be calculated. At regular intervals, the Boltzmann solver also updates and new coefficients are fed back into the ODEs for the ODE solver to work with. This results in the time evolution of species densities and electron temperatures.}
\label{fig:GlobalKin}
\end{figure}


\subsection{Vibrational Kinetics in Nitrogen Plasmas}

As mentioned previously, vibrational kinetics are important to include in plasma chemistries, due to their influence on the electron energy distribution function (EEDF) through population of vibrationally excited states by electron impact (electrons losing energy, \textbf{equation \ref{eqn:ElectronVibration}}), and superelastic collisions between electrons and vibrationally excited states (electrons gaining energy, \textbf{equation \ref{eqn:SuperelasticVibrational}}).
\begin{equation}
e^- + N_2(X,v=0) \rightarrow N_2(X,v=1) + e^-
\label{eqn:ElectronVibration}
\end{equation}
\begin{equation}
e^- + N_2(X,v=1) \rightarrow N_2(X,v=0) + e^-
\label{eqn:SuperelasticVibrational}
\end{equation}
Reactions involving vibrationally excited states also have an impact on nitrogen dissociation and gas heating, therefore, should be included to make the model as realistic as possible.
The first type of reactions to include are V-T reactions. These are particularly important for gas heating and are when vibrationally excited states collide with either ground state molecular or atomic nitrogen as follows:
\begin{equation}
N_2(X,v=n) + N_2 \rightarrow N_2(X,v=n-1) + N_2
\label{eqn:V-TN2}
\end{equation}
\begin{equation}
N_2(X,v=n) + N \rightarrow N_2(X,v=n-1) + N
\label{eqn:V-TN}
\end{equation}

The second type are V-V reactions, where two vibrationally excited states collide and vibrational energy is transferred between molecules, as follows:
\begin{equation}
N_2(X,v=n) + N_2 (X,v=w-1) \rightarrow N_2(X,v=w) + N_2(X,v=n-1)
\label{eqn:V-V}
\end{equation}

GlobalKin requires each heavy particle reaction to have a rate specified in Arrhenius form, while electron impact reaction rates require cross sections, which are built in to GlobalKin.

\subsubsection{Vibration-Vibration Reaction Rates}

Whilst most reaction rates can be found in literature, rates for vibration-vibration (V-V) reactions are not widely available, other than for n = 1 $\rightarrow$ 0 and the reverse reaction (n = 0 $\rightarrow$ 1).
Some rates are stated in Billing and Fisher, 1979 \cite{Billing1979vv}, however, in a recent paper from Pintassilgo and Guerra \cite{Pintassilgo2017modelling}, new rates were published for both V-V and V-T reactions, where $n = 1 \rightarrow 0$ (and the reverse reaction $n = 0 \rightarrow 1$.) 
Importantly, this paper also referenced equations for calculating rates from Alves \textit{et al}, 2012 \cite{Alves2012capacitively}, which means that further reaction rates, for different vibrational quantum numbers ($n$ and $w$) can be calculated.

Using the equations in \cite{Alves2012capacitively}, I have tried to replicate the rates published in \cite{Pintassilgo2017modelling}.
However, so far, I have been unable to replicate the rates.
As shown in \textbf{figure \ref{fig:paper_vs_equations}}, whilst the trend in the data matches, unfortunately, the absolute numbers are not replicable, with the calculated rates (solid lines) being consistently lower than those published in \cite{Pintassilgo2017modelling} (points).

I have tried reproducing this many times, both by hand and using Matlab scripts, and I have cross checked with different publications that give parts of the equations to check for any obvious typographical errors. 
%I have found one typo between two different papers, but changing that made it more wrong.
I have tried different temperatures, to see if maybe they have stated the wrong temperatures for the rates, but that doesn't help either.
Following discussion with Andrew Gibson, who is also unsure as to why there are discrepancies, I have emailed the author, Vasco Guerra with the issues I am having to see if he will be able to shed any light on the problem.
I think that the most likely issue is that either there is a typographical error in the Alves \textit{et al} equations, or that Pintassilgo and Guerra have used some form of correction factor to calculate the rates, and not included it in their publication. 

\begin{figure}
\centering
\includegraphics[width=0.7\textwidth]{Figures/paper_vs_equations}
\caption{V-V rates from Pintassilgo and Guerra, 2017 \cite{Pintassilgo2017modelling} shown by the points. V-V rates calculated from equations by Alves \textit{et al}, 2012 \cite{Alves2012capacitively} shown by the solid lines. Red shows rates for n=1 $\rightarrow$ 0 and w = 1 - 20, and blue shows the reverse reaction (n = 0 $\rightarrow$ 1).}
\label{fig:paper_vs_equations}
\end{figure}

%\subsubsection{Vibration-Translation Reaction Rates}
%
%Vibration-Translation reactions are also important reactions to include as they are one of the main gas heating mechanisms in nitrogen plasmas \cite{Pintassilgo2017modelling}. 
%These have been included in my chemistry set until now, with rates taken from Billing and Fisher, 1979 \cite{Billing1979vv}. 
%However, it was thought that these rates were too high, therefore a new method for calculating the rates has been presented by Guerra \textit{et al} \cite{Guerra2004kinetic} and again by Alves \textit{et al} \cite{Alves2012capacitively}.
%This is the method referenced by Pintassilgo and Guerra in their publication earlier this year \cite{Pintassilgo2017modelling}.
%However, as with the V-V reaction rates shown above, using the equations referenced, I have been unable to replicate the rates they present in a graph in their paper.
%The two sets of rates are shown in \textbf{figure \ref{fig:atomic_VT_comparison}}.
%Unfortunately, there are many orders of magnitude difference between the rates, which is not helpful :-(.
%
%Similarly to the V-V rate discrepancies, I have tried many things to fix the problem, but have so far been unsuccessful.
%
%\begin{figure}
%\centering
%\includegraphics[width=0.7\textwidth]{Figures/atomic_VT_comparison}
%\caption{VT rates comparison}
%\label{fig:atomic_VT_comparison}
%\end{figure}

%\subsubsection{Testing Vibrational Kinetics Rates}
%
%To test what difference the different rates have, it would be useful to be able to see what effects the different rates have on the VDF for each case, along with the atomic nitrogen density.
%However, there are issues with doing this as the moment.
%
%For V-T reactions, the rates shown in Pintassilgo suggest that VT reactions do not happen at an appreciable rate for quantum numbers below 22.
%My chemistry set only contains the first 17 vibrational states of ground state nitrogen, therefore, I would have to neglect the V-T reactions entirely... 
%For V-V reactions, it is hard to test them at this stage, because only having 1-0 and 0-1 skews the VDF anyway, and there are no other rates available that I have found for any other n values, therefore there is nothing to compare against.

\subsection{Pulsed Plasma Simulations}

Since any nitrogen or air plasmas will need to be pulsed (discussed further in \textbf{section \ref{subsec:exp_characterisation}}), I have investigated using GlobalKin to simulate pulsed plasmas by specifying a power cycle.
For this, I used a He/O$_2$ chemistry set, developed by Sandra Schr\"oter, and compared the density of the hydroxyl radical in both the pulsed and non-pulsed plasmas.

\textbf{Figure \ref{fig:pulsed_power}}, shows the different power densities over time in the pulsed and non-pulsed situations.
For both, the plasma geometry (3 x 1 x 1.1 cm channel) and maximum power (13.56 MHz, 60.61 Wcm$^{-3}$) is the same.
For the non-pulsed situation, the power density is 60.61 Wcm$^{-3}$ along the whole channel, whereas for the pulsed situation, the $\mu$second scale pulses have a rise time of 2.4 $\mu$s, on time of 20 $\mu$s, fall time of 3.1 $\mu$ and the period of the cycle is 100 $\mu$s.

\begin{figure}
\begin{subfigure}{0.5\textwidth}
\includegraphics[width=\textwidth]{Figures/pulse_power}
\caption{Power}
\label{subfig:pulsed_power}
\end{subfigure}
%\begin{subfigure}{0.5\textwidth}
%\includegraphics[width=\textwidth]{Figures/pulse_Tgas}
%\caption{Gas temperature}
%\end{subfigure}
\begin{subfigure}{0.5\textwidth}
\centering
\includegraphics[width=\textwidth]{Figures/pulse_OH}
\caption{OH Density}
\label{subfig:pulsed_OH}
\end{subfigure}
\caption{Figure showing comparisons for different plasma properties and hydroxyl (OH) density when the power applied to the same plasma channel is either pulsed, or not pulsed. Plasma channel is 3 x 1.1 x 0.1 cm. Power density is 60.61 Wcm$^{-3}$.}
\end{figure}

\textbf{Figure \ref{subfig:pulsed_power}} shows the difference in the power densities in the pulsed and non pulsed situation, and \textbf{figure \ref{subfig:pulsed_OH}} shows the densities of OH in the two situations over time.
It can be seen that the OH density falls during the off time, but has not depleted completely before the next pulse arrives, causing breakdown and OH production once more.
The densities of OH are much lower in the pulsed situation, presumably due to the reduced time for which electrons are energetic enough to cause reactions to happen.

\subsection{LTP-Induced Membrane Damage}

\subsubsection{Aims}
In collaboration with Paulina Dubiel, a biology summer student, we have begun to investigate the effects of LTP treatment on the membrane of \textit{E. coli} MG1655.
Using a membrane dye, FM4-64, the integrity of the bacterial membrane following LTP exposure can be investigated using fluorescence microscopy.

\subsubsection{Methods}

The plasma source used consists of two plane-parallel electrodes, with the channel enclosed by clear windows, forming a channel of 3 x 0.1 x 0.1 cm. 
One electrode is driven by a 13.56 MHz radio frequency (RF) generator via an impedence matching box, while the other is grounded. 
The applied voltage was measured using a voltage probe (between the matching box and plasma source) and oscilloscope.
The voltage was kept constant at 640 V.
The feed gas was 1 slm Helium, admixed with 5 sccm oxygen (0.5\%), controlled by mass flow controllers. 
The plasma jet was positioned so that the tip was 2 cm above the surface of the cell suspension.

\textit{E. coli} MG1655 were prepared in biology by Paulina Dubiel. 
They were grown to an OD600 of 2.7, then 1 mL of cell suspension was placed in a well of a 48 well plate for plasma treatment in YPI.
The depth of the liquid was approximately 0.5 cm.
Following exposure, cells were stained with FM4-64 dye (at 30 $\mu$M concentration) then put on microscope slides, covered with a coverslip, then sealed with nail varnish and viewed using fluorescence microscopy.

A picture of the treatment setup is shown in \textbf{figure \ref{fig:setup_pic}}.
The length of treatment was varied between 30 seconds and 120 seconds.
An untreated control was included. 
However for future experiments, further controls that will also be included are (1) cells treated only with gas flow and (2)
cells treated by power only (i.e. power on, but no gas flow so no plasma).
Also, in future experiments, the temperature of the plasma effluent will also be measured to make sure that isn't changing too much and causing effects of its own.

\subsubsection{Results}
Cells were treated with plasma for 30, 60, 90 or 120 seconds.
\textbf{Figure \ref{fig:results}} shows a representative microscopy image of a cell sample treated for 120 seconds.
Healthy cells with intact membranes should be bright red.
However, while there are many cells that are bright red, there are a few which show only a dim halo of fluorescence, suggesting that their membrane is damaged.
However, this is not certain and in the future, to help with determination of truly damaged cells, it has been suggested that we try an intracellular stain, that can only enter the cell if the membrane is damaged.
This way, it should give a clearer idea as to whether the cells are live, with a healthy membrane, or dead with membrane damage.

%TEMPERATURE????


%The temperature of the plasma effluent incident on the cells was measured using a thermocouple 2 cm below the end of the plasma channel.
%The temperature was $\approx$ X$^\circ$C \todo{Temperature for treatments}.
%Changing the applied voltage affected the temperature as shown in \textbf{figure \ref{fig:voltage_temp}}.



\begin{figure}
\centering
\includegraphics[width=0.6\textwidth, angle=270]{Figures/setup_pic}
\caption{Photograph showing the plasma jet and plate containing bacteria}
\label{fig:setup_pic}
\end{figure}

\begin{figure}
\centering
\includegraphics[width=0.5\textwidth]{Figures/120s_cells}
\caption{Image showing \textit{E. coli} stained with FM4-64 following 120 second treatment with He:O$_2$ LTP. Bright red cells indicate healthy intact membranes, whereas the dim red outlines of cells suggest that their membranes are damaged.}
\label{fig:results}
\end{figure}


\section{Future Directions}

\subsection{Simulations}
%\subsubsection{V-V and V-T reactions and rates}

Since finding the discrepancies between the calculated and presented V-V and V-T rates, I have emailed Vasco Guerra directly to see if he is able to offer any help on the differences seen.
Once this has been rectified, hopefully I should have a method for calculating the rates for each reaction for each vibrational quantum number.
This would be highly beneficial for developing the nitrogen chemistry set further.

In the mean time there are other things to work on in the chemistry set, for example, making sure it is realistic for atmospheric pressure plasmas.
%\subsubsection{Correcting for Pressure and Other things...}
Currently, the reaction set I am working from from \cite{Kutasi2016tuning}, is concerned with low pressure plasmas.
%However, my aim is to develop the model for atmospheric pressure plasmas.
In order to make it suitable for atmospheric pressure, I need to make sure that no reactions are missing that are particularly important at high pressures.
Once such class of reactions are three body reactions.
These are reactions that have three reactants, and are only really likely at higher pressures, due to there being enough chance that three particles can come together at the same time to react. 

Also important to start, is the model benchmarking process.
For this, atomic nitrogen can be measured in experimental plasmas to compare with the simulated data.
This will be discussed more in \textbf{section \ref{subsec:exp_characterisation}}.
%\todo{What are 3 body reactions? Are they the ones that require a third body to work? and that the pressure is important because of the probability of that third body being present?}.
%I also need to work on the ability of GlobalKin to calculate the time evolved gas temperature.
%This involves making sure that the correct enthalpies for reactions

%Also, need to fix temperature evolution. The model currently gives very weird looking temperature evolution data.
%This is all part of the requirement for benchmarking as there are other things that look weird too, but that can be investigated more by benchmarking. e.g. having real powers and geometry rather than guessing etc.


%\begin{itemize}
%\item Add in appropriate V-V and V-T reactions and rates
%\item Check about pressures - anything missing due to difference between low and atmospheric pressure?
%\item Start adding in oxygen species/reactions following benchmarking of model, or simultaneously using proper version control!!
%\end{itemize}

\subsubsection{Extending the Nitrogen Chemistry Set to a N$_2$:O$_2$ Chemistry Set}

Following the benchmarking of the nitrogen chemistry set, the next step is to move the chemistry set forward, from being just for nitrogen, to being for nitrogen and oxygen (i.e. air).
As with the nitrogen set, I will use the chemistry set provided by Vasco Guerra from \cite{Kutasi2016tuning}, as a starting point for reactions with only oxygen species, and ones involving both oxygen and nitrogen.

\subsection{Experimental Plasma Characterisation}
\label{subsec:exp_characterisation}

For both nitrogen plasmas (for model benchmarking), and for air plasmas, a source is required.
The plasma source being built has a parallel plate configuration, and will be powered by a nanosecond pulses.
A diagram of the source configuration is shown in \textbf{figure \ref{fig:plasma_diagram}}.
Pulsing plasmas allows for more fine tuning of certain properties, such as gas temperature and chemistry.
Important parameters of the pulsing are the rise and fall times, and the on time.

%The on time can then be adjusted to control the chemistry and give an extra level of control (on top of applied power/voltage etc).

Pulsing allows greater control of a plasma operating within a specific parameter range, such as at specific powers.
This is due to the ability to alter different pulse parameters, such as rise, on, fall, and off times as follows:
\begin{itemize}
\item Rise time - This is the time taken for the voltage to rise from 0 to its maximum amplitude. 
Ideally this should be as short as possible to give a well characterised, reproducible plasma breakdown.
\item On time - This is the time when the voltage stays at its maximum amplitude. 
It can be tailored to be long enough for reactions to take place sufficiently, but not too long that too much energy is transferred to heating the plasma. 
Heating is a particular problem with atmospheric pressure plasmas due to their being many collisions between electrons and neutrals, meaning that even though each collision only transfers a small amount of energy to the neutrals, there are enough collisions to heat the overall gas. 
Molecular plasmas also have this problem, because molecules are able to store lots of heat energy in their bonds in, for example, vibrationally excited states.
Therefore, the on time can be tailored so that the on time does not allow for significant heating of the plasma.
\item Fall time and off time - This is the time taken for the voltage to fall from maximum to minimum, and the time when the voltage is at the minimum, before the next pulse. The fall time should also be short. 
During the fall time and the off time between pulses, there is no electric field so electrons lose their energy and the plasma goes out. 
This means that short lived species, such as atoms, are also lost through, and only longer lived species remain. 
Following this, some longer lived species will remain until the next pulse, when the voltage rises again and the gas breaks down again.
\end{itemize}

Therefore, these plasmas will be powered by pulses, with short rise and fall times, and a suitable on time such that the plasma chemistry can progress, but the heating processes are not too great to heat the plasma excessively.
For this to work, the pulses will be on the nanosecond time scale.

%\textbf{Figure \ref{fig:plasma_source}} shows a diagram of the plasma source. \todo{Ask Jerome about the source....}

Once the plasma source is in operation, the aim is to measure the densities of different species using two-photon absorption laser-induced fluorescence (TALIF).
For benchmarking of the nitrogen model atomic nitrogen will be measured.
For this technique, two photons of a specific wavelength will excite ground state atomic nitrogen to an excited state.
The excited state will then decay down a more stable, lower energy state and emit a photon, which can then be detected.
The problem with this technique is a process called collisional quenching. 
This is when other species present in the plasma collide with the excited state and therefore, reduces its lifetime.
This is a problem if not all the quenching partners are known, so to counteract it, plasmas need to be diagnosed on very short timescales.
Therefore, picosecond laser pulses and very fast detection systems can be used to be able to detect the fluorescence before quenching becomes a problem.

%\subsubsection{Benchmarking of Nitrogen Model}
%
%To validate the nitrogen model, it is important to compare simulated densities with experimental measurements.
%Therefore, a source for pure nitrogen plasmas is required.
%The plasma source being built has a parallel plate configuration, and will be powered by a nanosecond pulses.
%Pulsing plasmas allows for more fine tuning of certain properties, such as gas temperature and chemistry.
%Important parameters of the pulsing are the rise and fall times, and the on time.
%
%The on time can then be adjusted to control the chemistry and give an extra level of control (on top of applied power/voltage etc).
%
%Pulsing allows greater control of a plasma operating within a specific parameter range, such as at specific powers.
%This is due to the ability to alter different pulse parameters, such as rise, on, fall, and off times as follows:
%\begin{itemize}
%\item Rise time - This is the time taken for the voltage to rise from 0 to its maximum amplitude. 
%Ideally this should be as short as possible to give a well characterised, reproducible plasma breakdown.
%\item On time - This is the time when the voltage stays at its maximum amplitude. 
%It can be tailored to be long enough for reactions to take place sufficiently, but not too long that too much energy is transferred to heating the plasma. 
%Heating is a particular problem with atmospheric pressure plasmas due to their being many collisions between electrons and neutrals, meaning that even though each collision only transfers a small amount of energy to the neutrals, there are enough collisions to heat the overall gas. 
%Molecular plasmas also have this problem, because molecules are able to store lots of heat energy in their bonds in, for example, vibrationally excited states.
%Therefore, the on time can be tailored so that the on time does not allow for significant heating of the plasma.
%\item Fall time and off time - The fall time should also be short. 
%During the fall time and the off time between pulses, there is no electric field so electrons lose their energy and the plasma goes out. 
%This means that short lived species, such as atoms, are also lost through, and only longer lived species remain. 
%Following this, some longer lived species will remain until the next pulse, when the voltage rises again and the gas breaks down again.
%\end{itemize}
%
%Therefore, the benchmarking nitrogen plasma will be powered by pulses, with short rise and fall times, and a suitable on time such that the plasma chemistry can progress, but the heating processes are not too great to heat the plasma excessively.
%For this to work, the pulses will be on the nanosecond time scale.
%
%\textbf{Figure \ref{fig:plasma_source}} shows a diagram of the plasma source. \todo{Ask Jerome about the source....}
%
%Once the plasma source is in operation, the aim is to measure atomic nitrogen density using two-photon absorption laser-induced fluorescence (TALIF).
%For this technique, two photons of a specific wavelength will excite ground state atomic nitrogen to an excited state.
%The excited state will then decay down a more stable, lower energy state and emit a photon, which can then be detected.
%The problem with this technique is a process called collisional quenching. 
%This is when other species present in the plasma collide with the excited state and therefore, reduces its lifetime.
%This is a problem if not all the quenching partners are known, so to counteract it, plasmas need to be diagnosed on very short timescales.
%Therefore, picosecond laser pulses and very fast detection systems can be used to be able to detect the fluorescence before quenching becomes a problem.
%



%\begin{itemize}
%\item Parallel plate configuration. Nanosecond pulses. Anything else electric-y? Rise time, on time and off time. Want short rise and fall and can tune the on time to control the chemistry. In the off, electrons decay very quickly and are lost but the longer lived species last longer. Short lived things are also lost eg atoms. Controls heat. So things heat up through electrons colliding with neutrals/heavy particles so by longer on time, more chance of collisions so even through small energy transfers, the neutrals heat up so everything heats up. Therefore, by keeping the pulses short, chemistry continues but the heating processes slow down/stop (kind of like reducing the power cos electrons won't totally lose all their energy probably) so everything doesn't heat up as much. 
%Molecules are also worse for heating because at atmospheric pressure there are a lot of collisions and a lot of energy can be stored by the molecules in internal degrees of freedom (eg excited states), therefore, need more energy to breakdown, and they can store more heat too, so everything gets hotter more.
%\item Picture of source? \todo{Take a picture of source!} \textbf{figure \ref{fig:plasma_source}}
%\item Take measurements of atomic nitrogen
%\item TALIF? Problems with this are quenching. So excited states are lost through collisions with quenching partners. For things like pure nitrogen, we can know the quenching partners (e.g. N$_2$) so can deal with that, but for things like air, we don't know all the quenching partners so it is a problem because we can't understand the behaviour of them all. However, the problem with quenching is the reduction in the lifetime of the excited state.
%So to counteract this, we have to diagnose plasma on very short timescales. Therefore, use picosecond laser pulses and very fast detectors, to be able to detect the LIF before quenching is a problem.
%Air is a further problem, because the timescales are even shorter, so the picosecond setup might not work.... but we'll worry about that when it happens...
%\end{itemize}

\begin{figure}
\centering
\includegraphics[width=0.8\textwidth]{Figures/plasma_diagram}
\caption{Plasma source}
\label{fig:plasma_diagram}
\end{figure}

%\begin{figure}
%\centering
%\includegraphics[width=\textwidth]{Figures/Harry}
%\caption{Haribo $<$3}
%\label{fig:plasma_source}
%\end{figure}



\subsection{Effects of LTP on Wound Healing}
%
%\begin{itemize}
%\item Determining ox stress levels - markers for oxidative stress and how to measure them. Death, normal stuff RNA stuff?
%\item Determining wound healing promotion - scratch test and if time allows, a 3D skin model, e.g. MatTek EpiDerm FT.
%\item Draw a graph that has extent of ox stress and healing promotion on y axes, and changing plasma property on x axis
%\item In the future, would be good to repeat this with bacteria, to see if the effects are similar
%\end{itemize}




\subsubsection{Human Skin and the Healing Response}

The skin has three layers.
From superficial to deep, these are the epidermis, the dermis and the hypodermis, each of which have a different cellular composition and function as follows: %\todo{Check where immune cells hang out...}
\begin{itemize}
\item The epidermis is anchored to the dermis by a basement membrane, on which are epidermal stem cells, which are responsible for the continuous production of new keratinocytes.
As keratinocytes develop, the move upwards in the epidermis, through different layers, becoming increasingly differentiated, until they reach the most superficial layer, the stratum corneum. 
By this layer, the keratinocytes are terminally differentiated and dead, and are surrounded by keratins, cross-linked proteins and lipids, which together make skin impermeable.
Over time, this top layer is lost and replaced by keratinocytes underneath \cite{Mancini2014micro}.
%The epidermis is made up of keratinocytes and is split from the dermis by a basement membrane. On the basement membrane, there are basal cells which produce the keratinocytes, melanocytes and sensory cells. Keratinocytes then move upwards from the basal layer, begin to produce keratin and then die. The surface of skin is, therefore, dead keratinocytes and keratin, which gives a waterproof barrier, and is sloughed off over time, to allow new keratinocytes up to the surface \toref{A skin paper!}
\item Dermis is composed of dense, irregular connective tissues, sweat glands, hair follicles and blood vessels \cite{Mancini2014micro}
\item Hypodermis consists mainly of fatty tissue to support the connective tissue above \cite{Mancini2014micro}
\end{itemize}

Following wounding, the skin commences a well-defined healing phases: (1) coagulation and haemostasis, (2) inflammation, (3) proliferation and re-epithelialisation, and (4) wound remodelling \cite{Velnar2009the}.
Evidence suggests that RONS such as NO \cite{Shekhter2005beneficial}, and electric fields \cite{Thakral2013electrical, Messerli2011extracellular} have beneficial effects on the wound healing process, therefore, these are proposed mechanisms for the acceleration in wound healing seen by LTP treatment.
For more detail, please refer to my 6 month TAP report.


%The skin has different layers. The following information about skin layers comes from: 
%
%\url{https://opentextbc.ca/anatomyandphysiology/chapter/5-1-layers-of-the-skin/}
%
%
%There is the epidermis, the dermis and the hypodermis.
%The epidermis it the most superficial layer, exposed to the environment, and is composed almost entirely of keratin-producing cells, keratinocytes.
%The epidermis is made up of different layers, namely from deep to superficial, the stratum basale, stratum spinosum, stratum granulosum and stratum corneum.
%All except the stratum basale is made of keratinocytes, whereas the stratum basale consists mainly of basal cells (which are keratinocyte precurosrs), but also contains Merkel cells (touch receptors) and melanocytes (melanin-producing cells).
%As basal cells divide and produce daughter keratinocytes, the keratinocytes are pushed up through the epidermal layers, producing increasing amounts of keratin and then dying, leaving begin the keratin and cell membranes, so that the most superficial layer consists only of dead keratinocytes and their keratin.
%The epidermis also contains Langerhan's cells, which are a skin-specific dendritic cell, which is able to phagocytose invading pathogens and debris.
%
%The dermis is below the epidermis and contains structures such as blood vessels, hair follicles and sweat glands, in dense, irregular connective tissue.
%Finally, the hypodermis consists mainly of loose connective tissue and fatty tissue.
%
%A real paper on skin structure \cite{Mancini2014microRNAs}...
%
%When considering wound treatments using LTP, depending on the type and severity of the wound, the skin cells that will be exposed will vary.
%However, since keratinocytes form the bulk of the epidermis, it is reasonable to start any investigations of LTP effects on skin cells in these cells.

\subsubsection{Aims of Biology Experiments}

For the time being, I have decided to focus my project on the wound healing promotion aspect of wound treatments, as opposed to bacterial killing.
This is particularly important for determining the safety of LTP treatment, as well as looking into the beneficial effects induced by plasma on wound healing.
To begin with, cell lines will be used, for example, keratinocyte cell lines. 
However, if time allows, then the project will hopefully be able to progress into using a more advanced model for wound healing, such as the EpiDerm FT model from MatTek Corporation \cite{MattekWebsite}, which is a fully stratified, full thickness skin model, which has been used previously for wound healing investigations.

The aims of the biology arm of this project are as follows:
\begin{enumerate}
\item Decide on a suitable end point for measurement following LTP treatment. In particular, I am interested in the oxidative stress induced in skin cells following LTP treatment, and in measuring factors that are up/downregulated for successful wound healing and how they are affected by LTP treatment.
\item Develop a suitable experiment that is easily reproducible so can be repeated many times for different types of plasma treatment. As the overall aim of the biology is to look at how different plasma compositions can affect biological outcomes, it is important to have an experiment that can be repeated many times and be robust enough to give consistently measurable, quantitative results.
\item Determine how different plasma treatments can affect the biology outcomes
\end{enumerate}

%In particular, it will be of interest to look at the differential effects of LTP treatment on both healthy skin/skin cells, as well as the "wounded" skin cells immediately at the wound edge.
%This is because, even though they are the same cell type, their immediate environment will be very different.
%A diagram of this concept is shown in \textbf{figure \ref{fig:wound_diagram}}.

%\begin{figure}
%\centering
%\includegraphics[width=0.5\textwidth]{Figures/wound_diagram}
%\caption{A wound...}
%\label{fig:wound_diagram}
%\end{figure}

\subsubsection{Detection of Oxidative Stress in Skin Cells}

At low levels, RONS are highly important for normal physiological cell function \cite{Fang2004antimicrobial, Thannickal2000reactive}.
However, if the concentration of RONS rises excessively for cells, then they can have toxic effects, for example, causing protein damage \cite{PhamHuy2008free}, lipid peroxidation \cite{Ayala2014lipid} and DNA damage \cite{Dizdaroglu2012oxidatively}, leading ultimately to cell death.

In order to detect oxidative stress induced by plasma treatment, a suitable end point to measure needs to be determined.
The overall aim is the find a marker that is able to respond sensitively to changes in the degree of oxidative stress being experienced by the cell.
This way, it is hoped that the marker can distinguish the degree of stress in cells across the whole scale from barely stressed, to almost being killed due to excessive stress.

It seems that there are at least three ways to go about measuring oxidative stress, as follows:
\begin{enumerate}
\item Monitoring cell death.
This is a very nonspecific measurement, though it gives a rough overall picture of how damaging a treatment is, looking at the percentage of a cell population that can survive.
\item Measuring products of RONS attack.
For example, lipid peroxidation products such as malondialdehyde (MDA) and 4-hydroxynonenal (4-HNE) \cite{Ayala2014lipid} and DNA damage products such as 8-OHdG. Fluorescent probes can be used to detect these products \cite{Ayala2014lipid, Joshi2011nonthermal, Joshi2010control}.
\item Measuring an endpoint at the DNA/RNA/miRNA/gene expression level.
For example, Schmidt \textit{et al}, 2015 discovered that plasma treatment has an effect on regulating the NRF-2 pathway, which is important in oxidative stress. They also found that the mRNA expression of several ROS-scavenger enzymes and peroxiredoxins were upregulated. Examples include GPX1, -3 and -5 \cite{Schmidt2015non}.
By measuring oxidative stress at this level, it should allow for better quantitation, and therefore, calibration between degree of oxidative stress and the level of the chosen marker.
To investigate these possibilities more, I will speak to Dmitris Lagos in biology, who is particularly interested in miRNA/RNA expression.

%miRNA in wound healing \cite{Riemondy2014not} - a review suggests wound healing promoted by miR-99 downregulation, and others may also be involved in wound healing. Downregulation of miR-199a-5p and miR-200b supports angiogenesis.
%miR-200c is involved in wound healing. Downregulation is required for proper healing as upregulation inhibits cell migration during wound repair  \cite{Aunin2017exploring}. 
%miR-200c is also upregulated by oxidative stress \cite{Magenta2011miR}. Yay.
%miR-25??
\end{enumerate}





%Oxidative stress is when cells/tissues are overwhelmed by oxidants. 
%Oxidants are things which cause oxidation.
%Free radical chain reactions normally produce numerous non-radical oxidants, such as hydrogen peroxide \toref{reference: https://hstalks.com/t/374/introduction-to-free-radicals-and-the-oxygen-parad/?biosci} .
%
%Oxidative stress can lead to induction of lipid peroxidation and modulation in the levels of antioxidant and drug-metabolising enzymes.
%ROS have also been demonstrated to induce some transcription factors, such as activating protein 1 (AP-1), and NF-$\kappa$B. 
%UVA is also involved in NF-$\kappa$B activation.
%Mitogen-activated protein kinase (MAPK) pathway is also a target of oxidative stress \cite{Bickers2006oxidative}.
%
%\cite{Nadal2011controlling} - paper talking about different sensors/transduction mechanisms/effectors but mainly in yeast and drosophilia
%
%\cite{Bickers2006oxidative} - another paper talking specifically about the skin and the role of oxidative stress in disease
%
%Kim2005role - something to do with MAPK in mouse skin following heat treatment
%
%Escuin-Ordinas2016cutaneous - something to do with MAPK, ERK and wound healing!





\subsubsection{Quantifying Wound Healing Promotion}

Alongside the quantification of oxidative stress induction, it would be good to investigate the effects of LTP/oxidative stress on factors that are crucial for wound healing.
As a start it would be good to use keratinocyte cell lines and investigate their activation/proliferation following wounding.
To do this, a scratch wound assay could be used.
For this, keratinocytes are grown into confluent layers and then a scratch is made across the plate to simulate the wound.
The plates can then be watched and imaged over time to determine the rate of "healing", i.e. how fast the gap closes.
To extend this assay further and to give more information, it would also be of use to look at gene expression/transcription in the keratinocytes to be able to correlate the healing rate with other factors related to wound healing.

%To quantify wound healing promotion, I propose investigation of the gene expression profile of the keratinocytes, in particular to look for activation through keratin type expression.
Upon wounding, keratinocytes must become activated in order to migrate into the wound site to fill the defect.
Firstly, they release proinflammatory cytokines and growth factors, such as interleukin 1 (IL-1), tumour necrosis factor $\alpha$ (TNF$\alpha$) and epidermal growth factor (EGF).
These stimuli then allow keratinocytes to become activated and begin expressing K6,16 and 17, which is essential for keratinocyte migration into the wound \cite{Pastar2014epithelialization}.
Transcription factors, such as Slug, also helps with initiating keratinocyte migration into the wound by helping to release them from their neighbours \cite{Pastar2014epithelialization, Savagner2005developmental}.
Factors such as these should all be measurable, and therefore, it would be interesting to see how they are affected by plasma treatment.

\subsubsection{Experimental Plan to Combine Oxidative Stress and Wound Healing Investigations}

The ultimate aim of the biology experiments is to be able to see if there are any correlations between wound healing rate (scratch test), healing factor regulation (RNA/gene expression analysis) and oxidative stress (RNA/miRNA analysis).

A proposed experiment plan is as follows:

\begin{enumerate}
\item Keratinocytes will be prepared in plates suitable for the scratch test assay. 
Each scratch test assay will be done in duplicate so that one can be monitored to determine wound healing rate, and from the other, keratinocytes can be taken for qPCR analysis.
The reason for using a scratched plate for the gene expression analysis, is so that the plates are equivalent and the cells have the same "injured" environment, and will be experiencing the same stresses.
This will also be controlled for carefully by doing multiple scratch test assays under the same conditions and checking that they behave similarly.
\item Taking the first scratch plate, it will be imaged repeatedly at defined time points, and the gap will be measured. The decrease in gap size over time can then be measured and the healing rate determined.
\item Taking the second treated plate, keratinocytes will be isolated and their DNA/RNA extracted.
The appropriate primers will then be used to carry out qPCR to monitor gene expression in the cells.
\item The aim of the analysis is to be able to determine any correlations between healing rate, oxidative stress level and healing gene expression. This way, it is hoped that it will be possible to find the optimum plasma condition for optimum wound healing and least oxidative stress induction.
\item This process can then be repeated multiple times for multiple different plasma conditions such as changes in maximum power, power pulse shape and humidity etc.
\end{enumerate}

If time allows, there are a number of options to follow up this experiment.
Firstly, it would be really interesting to translate this assay into a more advanced skin model.
For example, it is possible to purchase full thickness, fully stratified \textit{in vitro} skin.
MatTek EpiDerm FT \cite{MattekWebsite} is one such model, and it has been used before for wound healing assays.
Therefore, it would be good to carry out a similar wound healing assay to the scratch test above, on the full thickness skin model and then be able to investigate the gene expressions of keratinocytes, or other skin cell types, in a more realistic environment.
It would also allow investigation of the effects on deeper tissue, for example the dermis, which contains other types of cells and structures which may be affected by the LTP treatment.
However, these models are more expensive than cell lines, therefore, it would be a good idea to use cell lines to start with as a proof of concept.


%\begin{itemize}
%\item Need a skin model - MatTek EpiDerm FT \cite{MattekWebsite} is a possibility. They have done wound healing assays before, cool ones using fluorescence etc.
%\item Simple scratch assay - monolayer of keratinocytes in a petri dish, scratch then and see how long it takes for them to re-grow to confluence. Could be interesting to look at different things alongside this. Ie, have two plates scratched identically - watch one regrow and take the other to look at gene/miRNA expression/oxidative stress/markers of wound healing. Type of keratinocyte changes on activation \cite{Pastar2014epithelialization} and transcription factor induction (e.g. slug \cite{Savagner2005developmental}).
%\item Do keratinocytes need to be wounded to be able to induce the wound healing promotion?! Or is the role of oxidative stress different in wounded and healthy keratinocytes?? \textbf{Figure \ref{fig:wound_diagram}}
%\end{itemize}

\subsection{COST jet comparison}

COST microplasma jets are designed to be reference plasma sources that behave comparably, with a particular focus on biomedical applications \cite{Golda2016concepts}.
Recently 4 jets have been compared in the lab, in terms of power, temperature and optical emission.
The next step is to determine if all the jets also have comparable abilities to kill bacteria.
To investigate this, E.coli MG1655 on agar plates will be exposed to the plasma jets and the area of the plates where bacteria has been killed will be analysed, the so-called killing zone.
This will be a good opportunity to work with bacteria and will help with my project.

%\begin{itemize}
%\item COST microplasma jets are designed to be reference jets, for biomedical applications \todo{Check if for bio stuff}
%\item Have 4 jets being compared in the lab. So far all compare well in terms of power, temperature and emission. Now want to check biology interactions
%\item Aim is to determine whether they al have equivalent ability to kill bacteria, therefore the killing zone of each of the jets will be determined, by treating agar plates of E.coli MG1655 with each jet, then the kill zone established.
%\item Bacteria will be grown, then plated onto agar plates, before being exposed to the plasma jet. 
%The plates will then be incubated, and colonies counted and pictures taken, to determine the extent of the area of bacteria that has been killed by the plasma.
%\end{itemize}
 

\section{Immediate Plans}
In the immediate future the plan is to:
\begin{itemize}
\item Begin benchmarking of the nitrogen model using a pure nitrogen plasma
\item When Vasco replies, finish sorting out the vibrational kinetics in the nitrogen chemistry set
\item Learn how to do tissue culture of keratinocytes
\item Learn to culture bacteria for COST jet comparison
\item Meet with the appropriate people to be able to find possible markers of both oxidative stress and wound healing to measure in future biology experiments
\end{itemize}



\section{Project Outlook}
The project has 3 main aims that can run in parallel. (1) Simulations (2) Experimental plasma characterisation and (3) Biology.

To progress with the simulations, the nitrogen model requires benchmarking then can be extended to contain oxygen species and reactions.
Subsequent benchmarking of the air plasma simulations can then run alongside the diagnostics of the experimental air plasma designed for treating cells in the biology experiments.
To progress with experimental characterisation, the plasma source needs to be finished so that nitrogen and air plasma diagnostics can begin.
At the same time, the plasma source can be used to treat cells to start developing the biology experiment.
Once the biology experiment has been sorted, then the biology outcome of different plasma compositions can be tested.

















%\section{Oxidative Stress}
%Oxygen derived species - $\cdot$OH, O$_2^-\cdot$ and H$_2$O$_2$. However, O$_2^-\cdot$ and H$_2$O$_2$ have a low chemical reactivity, and do not readily react with biological molecules such as DNA, proteins and lipids. On the other hand, $\cdot$OH is very highly reactive, with reaction rates at or near diffusion-controlled rates for reactions with DNA \cite{Dizdaroglu2012oxidatively}.
%
%\subsection{DNA Damage}
%
%
%\section{Detecting Oxidative Stress - Bacteria}
%Different ways of detecting oxidative stress in bactera.
%In general, things to look for are cell death, membrane damage, DNA damage, lipid peroxidation and functional changes in proteins (e.g. enzyme deactivation).
%
%\subsection{Cell death}
%Measuring cell death can be done by measuring population level cell death, e.g through looking at zones of inhibition etc or at cellular level, e.g. using exclusion dyes.
%
%\subsection{Lipid Peroxidation}
%\begin{itemize}
%\item Lipid - large molecules made from smaller units of fatty acids and glycerol 
%\item Fatty acid - carboxylic acid consisting of a hydrocarbon chain and a terminal carboxyl group.
%\end{itemize}
%
%Radicals can attack polyunsaturated fatty acids and initiate lipid peroxidation in membranes.
%This can have significant effects on membrane fluidity and membrane-bound proteins.
%As a result of lipid peroxidation, aldehydes are often formed, which are long lived and able to function as "toxic second messenger" molecules, which can cause further damage to cells, in particular proteins \cite{Cabiscol2000oxidative}.
%These products include malonaldehyde (MDA) and 4-hydroxynonenal (HNE), which are particularly mutagenic and toxic, respectively \cite{Ayala2014lipid}.
%
%\subsection{DNA damage}
%
%\section{Methods}
%
%\begin{itemize}
%\item XTT assay. Metabolic activity
%
%Paper talking about oxidative stress in bacteria \cite{Cabiscol2000oxidative}. Talks about attack on lipids, proteins and DNA.
%
%\cite{Ezraty2017oxidative} is a paper with lots of biochemistry-y stuff in. Haven't read it - think it is a bit too in depth all about the mechanisms of protein damage, in particular sulphur containing side chains of proteins.
%
%\cite{Alkawareek2014potential} Paper looked at LTP interactions with bacteria, in particular bacterial killing/metabolism XTT assay (E. Coli, P. aeruginosa, B. cereus and MRSA), plasmid DNA damage (single and double strand breaks by looking whether the plasma was supercoiled (undamaged), open circular (single strand break) or linear (double strand break)), protein function (proteinase K's function was watched to see if the plasma was altering the structure in some way to prevent catalytic activity (it is pretty robust to changes in temp and pH therefore decrease in activity likely to be due to structural changes)), lipid peroxidation (MDA measurement in E. Coli) and membrane permeability (by measuring ATP leakage from E. Coli).
%\end{itemize}
%
%Assays:
%\begin{enumerate}
%\item \url{https://www.caymanchem.com/product/589320} - DNA/RNA Oxidative Damage ELISA Kit. Looks for 8-hydroxy-2'-deoxyguanosine from DNA, 8-hydroxyguanosine from RNA, and 8-hydroxyguanine from either DNA or RNA.
%\item
%\end{enumerate}
%%\section{}
%%\subsection{}
%
%\section{Detecting Oxidative Stress - Skin Cells}
%
%
%
%\section{Skin Models for Wound Healing Assays}
%Skin explants (people) can be cultured (as an organ culture) for up to 2 weeks at the air/liquid interface \cite{Gottrup2000models}.
%\subsection{Porcine Skin}
%
%\subsection{Hyaluronic Acid}
%
%\section{Cancer Things}
%\subsection{Radiotherapy}
%Radiation therapy can either be photon or particle, radiation.
%Photon radiation uses gamma ray or x ray beams focussed on the tumour, whereas particle radiation uses electrons, neutrons or protons for the same purpose - energy deposition into the tumour.
%By targeting the tumour with these beams, they deposit energy in the tumour cells, and cause ionisation of particles in the cells.
%This then leads to radical formation etc which causes damage to cancer cell DNA and prohibits their replication.
%Healthy cells are generally better at repairing DNA defects, meaning that if doses are delivered that cause sub-lethal levels of damage to healthy cells, then the healthy cells have time to repair before the next dose (as they divide much more slowly than cancer cells, so have longer to repair damage), whereas, cancer cells don't and therefore die.
%This gives specificity of a fashion.
%Damage to DNA occurs both directly (ie, the radiation directly damages the DNA), or indirectly, through formation of radical species which then attack the DNA. \cite{Baskar2012cancer}
%
%
%\subsection{Immunotherapy}
%\subsection{Photodynamic therapy}



\scriptsize
\bibliographystyle{ieeetr}
\bibliography{/Users/hld523/Bibliography/MyPapers}
\end{document}  